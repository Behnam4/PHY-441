\documentclass[fleqn]{article}
\oddsidemargin 0.0in
\textwidth 6.0in
\thispagestyle{empty}
\usepackage{import}
\usepackage{amsmath}
\usepackage{graphicx}
\usepackage{flexisym}
\usepackage{calligra}
\usepackage{amssymb}
\usepackage{bigints} 
\usepackage[english]{babel}
\usepackage[utf8x]{inputenc}
\usepackage{float}
\usepackage[colorinlistoftodos]{todonotes}


\DeclareMathAlphabet{\mathcalligra}{T1}{calligra}{m}{n}
\DeclareFontShape{T1}{calligra}{m}{n}{<->s*[2.2]callig15}{}
\newcommand{\scriptr}{\mathcalligra{r}\,}
\newcommand{\boldscriptr}{\pmb{\mathcalligra{r}}\,}

\definecolor{hwColor}{HTML}{442020}

\begin{document}

  \begin{titlepage}

    \newcommand{\HRule}{\rule{\linewidth}{0.5mm}}

    \center

    \begin{center}
      \includegraphics[height=11cm, width=11cm]{asu.png}
    \end{center}

    \vline

    \textsc{\LARGE Statistical/Thermal Physics}\\[1.5cm]

    \HRule \\[0.5cm]
    { \huge \bfseries Homework 3}\\[0.4cm] 
    \HRule \\[1.0cm]

    \textbf{Behnam Amiri}

    \bigbreak

    \textbf{Prof: Michael Treacy}

    \bigbreak

    \textbf{{\large \today}\\[2cm]}

    \vfill

  \end{titlepage}

  From the Schroeder book, do problems 1.46, 1.47, 1.54, 2.1, 2.3 and 2.10 
  (the last two involve a computer; Python, MatLab, C/C++, fortran etc. are great, 
  but Excel will work too). This homework is due Wednesday, February 3.

  \begin{enumerate}
    \item \textbf{1.46} Measured heat capacities of solids and liquids are almost always at constant  pressure,
    not constant volume. To see why, estimate the pressure needed to keep $V$ fixed as $T$ increases, as follows

    \begin{enumerate}
      \item First imagine slightly increasing the temperature of a material...

        % \textcolor{hwColor}{
        %   \\
        % }


      \item Now imagine slightly compressing the material, holding its temperature fixed...

        % \textcolor{hwColor}{
        %   \\
        % }

      \item Finally imagine that you compress the material just enough in part (b) to offset...

        % \textcolor{hwColor}{
        %   \\
        % }


      \item Compute $\beta, \kappa_T,$ and $\bigg( \dfrac{\partial P}{\partial T} \bigg)_V$ for an ideal gas...

        % \textcolor{hwColor}{
        %   \\
        % }

      \item For water at $25^{\circ}, ~ \beta=2.57 \times 10^{-4} ~ K^{-1}$ and $\kappa_T=4.52 \times 10^{-10} ~ Pa^{-1}$...

        % \textcolor{hwColor}{
        %   \\
        % }

    \end{enumerate}
    
    
    \item \textbf{1.47} Your $200-g$ cup of tea is boiling-hot. About how much ice should you add to bring 
    it down to a comfortable sipping temperature of $65^{\circ} ~ C$? (Assume that the ice is initially 
    at $-15^{\circ} ~ C$. The specific heat capacity of ice is $0.5 ~ cal/g.^{\circ}C.$) 


    \item \textbf{1.54} A $60-kg$ hiker wishes to climb to the summit of Mt.Ogden an ascent of $5000$ vertical
    feet (1500 m).
    \begin{enumerate}
      \item Assuming that she is $25\%$ efficient at converting chemical energy...

        % \textcolor{hwColor}{
        %   \\
        % }

      \item As the hiker climbs the mountain, three-quarters of the energy from...

        % \textcolor{hwColor}{
        %   \\
        % }

      \item In fact, the extra energy does not warm the hiker's body significantly; instead, it goes (mostly)...

        % \textcolor{hwColor}{
        %   \\
        % }

    \end{enumerate}


    \item \textbf{2.1} Suppose you flip four fair coins.
    \begin{enumerate}
      \item Make a list of all the possible outcomes, as Table 2.1.

        % \textcolor{hwColor}{
        %   \\
        % }

      \item Make a list of all different "macrostates" and their probabilities.

        % \textcolor{hwColor}{
        %   \\
        % }

      \item Compute the multiplicity of each macrostate usuing the combinatorial formula 2.6, and check 
      that these results agree with what you got by bruteforce counting.

        % \textcolor{hwColor}{
        %   \\
        % }
      
    \end{enumerate}


    \item \textbf{2.3} Suppose you flip 50 fair coins.
    \begin{enumerate}
      \item How many possible outcomes (microstates) are there?

        % \textcolor{hwColor}{
        %   \\
        % }

      \item How many ways are there of getting exacty 25 heads and 25 tails?

        % \textcolor{hwColor}{
        %   \\
        % }

      \item What is the probability of getting exactly 25 heads and 25 tails?

        % \textcolor{hwColor}{
        %   \\
        % }

      \item What is the probability of getting exactly 30 heads and 20 tails?
      
        % \textcolor{hwColor}{
        %   \\
        % }

      \item What is the probability of getting exactly 40 heads and 10 tails?

        % \textcolor{hwColor}{
        %   \\
        % }

      \item What is the probability of getting 50 heads and no tails?

        % \textcolor{hwColor}{
        %   \\
        % }

      \item Plot a graph of the probability of getting n heads, as a function of n.

        % \textcolor{hwColor}{
        %   \\
        % }

    \end{enumerate}

    \item \textbf{2.10} Use a computer to produce a table and graph, like those in this section, for the case
    where one Einstein solid contains 200 oscillators, the other contains 100 oscillator, and there are 
    100 units of energy in total. What is the most probable macrostate, and what is its probability? What
    is the least probable macrostate, and what is its probability?

        % \textcolor{hwColor}{
        %   \\
        % }

  \end{enumerate}

\end{document}
