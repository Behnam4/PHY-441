\documentclass[fleqn]{article}
\oddsidemargin 0.0in
\textwidth 6.0in
\thispagestyle{empty}
\usepackage{import}
\usepackage{amsmath}
\usepackage{graphicx}
\usepackage{flexisym}
\usepackage{calligra}
\usepackage{amssymb}
\usepackage{bigints} 
\usepackage[english]{babel}
\usepackage[utf8x]{inputenc}
\usepackage{float}
\usepackage[colorinlistoftodos]{todonotes}


\DeclareMathAlphabet{\mathcalligra}{T1}{calligra}{m}{n}
\DeclareFontShape{T1}{calligra}{m}{n}{<->s*[2.2]callig15}{}
\newcommand{\scriptr}{\mathcalligra{r}\,}
\newcommand{\boldscriptr}{\pmb{\mathcalligra{r}}\,}

\definecolor{hwColor}{HTML}{442020}

\begin{document}

  \begin{titlepage}

    \newcommand{\HRule}{\rule{\linewidth}{0.5mm}}

    \center

    \begin{center}
      \includegraphics[height=11cm, width=11cm]{asu.png}
    \end{center}

    \vline

    \textsc{\LARGE Statistical/Thermal Physics}\\[1.5cm]

    \HRule \\[0.5cm]
    { \huge \bfseries Quiz 10}\\[0.4cm] 
    \HRule \\[1.0cm]

    \textbf{Behnam Amiri}

    \bigbreak

    \textbf{Prof: Michael Treacy}

    \bigbreak

    \textbf{{\large \today}\\[2cm]}

    \vfill

  \end{titlepage}

  By signing my name, I am promising that I did this quiz on my own without any outside help.

  \vspace{0.5cm}

  Name: \textbf{Behnam Amiri}

  \vspace{1cm}

  \textbf{Give all answers to 3 significant figures.}

  \begin{enumerate}
    \item A system has $10^6$ potential wells, each with quantized energy levels $E_n=n \epsilon ~ (n=0,1,2,...)$
    and $\epsilon=0.01 ~ eV$. There are no degenerate states. The system is at $290 ~ K$ (so $k_B T=0.025 ~ eV$).
      \begin{enumerate}
        \item (4 points) Evaluate, numerically, the partition function.

          \textcolor{hwColor}{
            \\
            In the Boltzmann Distribution lecture, we learned that $Z=\sum\limits_{i} e^{-\lambda_2 E_i}$. For 
            finding the value of $\lambda$, we need to look at the energy of the system. We are considering a 
            thermodynamics system at a temperature $T$. By doing some mathematical reasoning, $\lambda_2$ was 
            found as $\lambda_2=\dfrac{1}{K_B ~ T}$.
            Since $\sum\limits_{i} \dfrac{n^*_i}{N}=1$ then $Z=\sum\limits_{i} exp\bigg( -\dfrac{E_i}{K_B T} \bigg)$.
            \\
            \\
            With the help of Calculus, we can rewrite the above series as
            \\
            \\
            $
              \sum\limits_{n=0} exp\bigg( -\dfrac{E_i}{K_B T} \bigg)=\dfrac{1}{1-e^{\epsilon/K_B T}}
              =\dfrac{e^{\epsilon/K_B T}}{e^{\epsilon/K_B T}-1}, ~~~~ (\text{Geometric serie})
              \\
              \\
              \\
              \therefore ~~~ \boxed{Z=\dfrac{e^{\epsilon/K_B T}}{e^{\epsilon/K_B T}-1}}
              \\
              \\
              \\
              Z=\dfrac{
                e^{0.01 \times \dfrac{1}{0.025}}
              }{
                e^{0.01 \times \dfrac{1}{0.025}}-1
              }
              =\dfrac{
                e^{0.4}
              }{
                e^{0.4}-1
              }
              \Longrightarrow
              ~~~ \boxed{
                Z \approxeq 3.033 ~ eV
              }
              \\
              \\
            $
          }

        \item (4 points) What is the average energy of the system (in $eV$)?

          \textcolor{hwColor}{
            \\
            The average energy is:
            \\
            \\
            $
              \langle E \rangle=\dfrac{1}{Z} \sum\limits_{s} E(s) e^{-\beta E(s)}, ~~~~ \beta=\dfrac{1}{K_B T}
              \\
              \\
            $
            Also, from the lecture we learned that the average energy can be found directly from the partition function
            by differentiating it.
            \\
            \\
            $
              \langle E \rangle=-\dfrac{1}{Z} \dfrac{\partial Z}{\partial \beta}
              \\
              \\
              \\
              \dfrac{\partial Z}{\partial \beta}
              =\dfrac{\partial}{\partial \beta} \bigg( \dfrac{1}{1-e^{\epsilon \beta}} \bigg)
              =\dfrac{0-(-\epsilon e^{\epsilon \beta})}{(1-e^{\epsilon \beta})^2}
              \\
              \\
              \\
              \therefore ~~~ \boxed{
                \langle E \rangle=-\dfrac{1}{Z} \dfrac{\epsilon e^{\epsilon \beta}}{(1-e^{\epsilon \beta})^2}, ~~~~ \beta=\dfrac{1}{0.025}=40
              } ~~~~ \checkmark
              \\
              \\
              \\
              \langle E \rangle=-\dfrac{1}{3.033} \dfrac{0.01 e^{ 0.01 \times 40 }}{(1-e^{0.01 \times 40})^2}
              \\
              \\
              \\
              \Longrightarrow \langle E \rangle \approxeq -0.02033 ~ eV
              \\
            $
          }

        \item (4 points) What is the heat capacity of the system (in $eV/K$)?

          \textcolor{hwColor}{
            \\
            $
              U=-\dfrac{\partial}{\partial \beta} \bigg( \ln(\dfrac{1}{1-e^{-\epsilon \beta}}) \bigg)
              =-(1-e^{-\beta \epsilon}) \dfrac{d}{d\beta} \bigg( \dfrac{1}{1-e^{-\beta \epsilon}} \bigg)
              \\
              \\
              \\
              =-(1-e^{-\beta \epsilon}) \dfrac{\epsilon e^{\beta \epsilon}}{(1-e^{-\beta \epsilon})^2}
              \\
              \\
              \\
              \therefore ~~~ \boxed{U=\dfrac{\epsilon e^{-\beta \epsilon}}{1-e^{-\beta \epsilon}}}
              \\
              \\
              \\
              \beta=\dfrac{1}{K_B T} \Longrightarrow \dfrac{\partial \beta}{\partial T}=-\dfrac{1}{K_B T^2} ~~~~ \checkmark
              \\
              \\
              \\
              \begin{cases}
                \dfrac{\partial}{\partial T} e^{-\epsilon/K_B T}=e^{-\epsilon/K_B T} \dfrac{-\epsilon}{K_B T^2}
                \\
                \\
                \dfrac{\partial}{\partial T} (1-e^{-\epsilon/K_B T})=e^{-\epsilon/K_B T} \dfrac{\epsilon}{K_B T^2}
              \end{cases}
              \\
              \\
              \\
              C_V=\bigg( \dfrac{\partial U}{\partial T} \bigg)_V
              =\dfrac{\partial}{\partial T} \left[
                \dfrac{
                  \epsilon e^{-\epsilon/K_B T}
                }{
                  1-e^{-\epsilon/K_B T}
                }
              \right]
              =\epsilon \left[
                \dfrac{
                  e^{-\epsilon/K_B T} \dfrac{-\epsilon}{K_B T^2} \bigg( 1-e^{-\epsilon/K_B T} \bigg)-e^{-\epsilon/K_B T} e^{-\epsilon/K_B T} \dfrac{\epsilon}{K_B T^2}
                }{
                  \bigg( 1-e^{-\epsilon/K_B T} \bigg)^2
                }
              \right]
              \\
              \\
              \\
              =\dfrac{\epsilon^2}{K_B T^2} \left[
                \dfrac{
                  -e^{-\epsilon/K_B T} (1-e^{-\epsilon/K_B T})-e^{-2 \epsilon/K_B T}
                }{
                  \bigg( 1-e^{-\epsilon/K_B T} \bigg)^2
                }
              \right], ~~~~~ \dfrac{\epsilon^2}{K_B T^2}=\dfrac{\epsilon}{K_B T} \dfrac{\epsilon}{T}
              =\dfrac{0.01}{0.025} \dfrac{0.01}{290} \approxeq 1.4 \times 10^{-5}
              \\
              \\
              \\
              C_V=\bigg( 1.4 \times 10^{-5} \bigg) \left[
                \dfrac{
                  -e^{-0.01/0.025} (1-e^{-0.01/0.025})-e^{-0.02/0.025}
                }{
                  \bigg( 1-e^{-0.01/0.025} \bigg)^2
                }
              \right]
              \\
              \\
              \\
              \therefore ~~~ \boxed{
                |C_V|=8.63 \times 10^{-5}
              } ~~~~ \checkmark
              \\
              \\
            $
            Note that the given system has $10^6$ potential wells, hence;
            \\
            \\
            $
              \boxed{
                C_{V_{Total}}=10^6 \times 8.63 \times 10^{-5}=86.3 ~ eV/K
              } ~~~~ \checkmark
              \\
              \\
            $
          }

      \end{enumerate}
    
    \item For the $O_2$ molecules in the air of your room (at $300 ~ K)$:
      \begin{enumerate}
        \item (4 points) Calculate their most probable speed (in $m/s$);

          \textcolor{hwColor}{
            \\
            $
              v_p=\sqrt{\dfrac{2 K T}{m}}
              =\sqrt{\dfrac{2 \times 1.381 \times 10^{-23} ~ J/K \times 300 ~ K}{5.3 \times 10^{-26} ~ kg}}
              \\
              \\
              \\
              \therefore ~~~ \boxed{
                v_p=395.398 ~ m/s
              } ~~~~ \checkmark
              \\
              \\
            $
          }

        \item (4 points) Calculate their average speed (in $m/s$);

          \textcolor{hwColor}{
            \\
            $
              \bar{v}=\sqrt{\dfrac{8}{\pi} \dfrac{ K T }{m}}
              =\sqrt{\dfrac{8}{\pi} \dfrac{1.381 \times 10^{-23} ~ J/K \times 300 ~ K}{5.3 \times 10^{-26} ~ kg}}
              \\
              \\
              \\
              \therefore ~~~ \boxed{
                \bar{v}=  446.1589 ~ m/s
              } ~~~~ \checkmark
              \\
              \\
            $
          }

        \item (4 points) Calculate their r.m.s. speed (in $m/s$).

          \textcolor{hwColor}{
            \\
            $
              v_{rms}=\sqrt{\dfrac{3 K T }{m}}
              =\sqrt{\dfrac{3 \times 1.381 \times 10^{-23} ~ J/K \times 300 ~ K}{5.3 \times 10^{-26} ~ kg}}
              \\
              \\
              \\
              \therefore ~~~ \boxed{
                v_{rms}=484.2617 ~ m/s
              } ~~~~ \checkmark
              \\
              \\
            $
          }

      \end{enumerate}

    \item The ratio of $O_2$ to $N_2$ molecules in air is $N_O/ N_N \approx 21/79.$
    The formula for the number of collisions per second per unit area, $\dot{N}$ , of an ideal gas is
    $$
      \dot{N}=\dfrac{1}{4} ~ n \langle v \rangle,
    $$
    where $n$ is the number of molecules per unit volume and $\langle v \rangle$ is the average molecular speed.
    Since the average speeds of $O_2$ and $N_2$ molecules are different, the ratio of the rates of arrival
    of $O_2$ and $N_2$ molecules at a surface, $\dot{N}_O/\dot{N}_N$, will be different to their occurrence ratio in air,
    $N_O/ N_N$, and we have
    $$
      \dfrac{\dot{N}_O}{\dot{N}_N}=\alpha \dfrac{N_O}{N_N}, ~~~~ \alpha \neq 1.
    $$
    (6 points) Calculate $\alpha$.

        % \textcolor{hwColor}{
        %   \\
        % }

  \end{enumerate}

\end{document}
\textbf{(10 points)}