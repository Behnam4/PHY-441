\documentclass[fleqn]{article}
\oddsidemargin 0.0in
\textwidth 6.0in
\thispagestyle{empty}
\usepackage{import}
\usepackage{amsmath}
\usepackage{graphicx}
\usepackage{flexisym}
\usepackage{calligra}
\usepackage{amssymb}
\usepackage{bigints} 
\usepackage[english]{babel}
\usepackage[utf8x]{inputenc}
\usepackage{float}
\usepackage[colorinlistoftodos]{todonotes}


\DeclareMathAlphabet{\mathcalligra}{T1}{calligra}{m}{n}
\DeclareFontShape{T1}{calligra}{m}{n}{<->s*[2.2]callig15}{}
\newcommand{\scriptr}{\mathcalligra{r}\,}
\newcommand{\boldscriptr}{\pmb{\mathcalligra{r}}\,}

\definecolor{hwColor}{HTML}{442020}

\begin{document}

  \begin{titlepage}

    \newcommand{\HRule}{\rule{\linewidth}{0.5mm}}

    \center

    \begin{center}
      \includegraphics[height=11cm, width=11cm]{asu.png}
    \end{center}

    \vline

    \textsc{\LARGE Statistical/Thermal Physics}\\[1.5cm]

    \HRule \\[0.5cm]
    { \huge \bfseries Quiz 10}\\[0.4cm] 
    \HRule \\[1.0cm]

    \textbf{Behnam Amiri}

    \bigbreak

    \textbf{Prof: Michael Treacy}

    \bigbreak

    \textbf{{\large \today}\\[2cm]}

    \vfill

  \end{titlepage}

  By signing my name, I am promising that I did this quiz on my own without any outside help.

  \vspace{0.5cm}

  Name: \textbf{Behnam Amiri}

  \vspace{1cm}

  \textbf{Give all answers to 3 significant figures.}

  \begin{enumerate}
    \item A system has $10^6$ potential wells, each with quantized energy levels $E_n=n \epsilon ~ (n=0,1,2,...)$
    and $\epsilon=0.01 ~ eV$. There are no degenerate states. The system is at $290 ~ K$ (so $k_B T=0.025 ~ eV$).
      \begin{enumerate}
        \item (4 points) Evaluate, numerically, the partition function.


        \item (4 points) What is the average energy of the system (in $eV$)?


        \item (4 points) What is the heat capacity of the system (in $eV/K$)?
      \end{enumerate}
    
    \item For the $O_2$ molecules in the air of your room (at $300 ~ K)$:
      \begin{enumerate}
        \item (4 points) Calculate their most probable speed (in $m/s$);

        \item (4 points) Calculate their average speed (in $m/s$);

        \item (4 points) Calculate their r.m.s. speed (in $m/s$).

      \end{enumerate}

    \item The ratio of $O_2$ to $N_2$ molecules in air is $N_O/ N_N \approx 21/79.$
    The formula for the number of collisions per second per unit area, $\dot{N}$ , of an ideal gas is
    $$
      \dot{N}=\dfrac{1}{4} ~ n \langle v \rangle,
    $$
    where $n$ is the number of molecules per unit volume and $\langle v \rangle$ is the average molecular speed.
    Since the average speeds of $O_2$ and $N_2$ molecules are different, the ratio of the rates of arrival
    of $O_2$ and $N_2$ molecules at a surface, $\dot{N}_O/\dot{N}_N$, will be different to their occurrence ratio in air,
    $N_O/ N_N$, and we have
    $$
      \dfrac{\dot{N}_O}{\dot{N}_N}=\alpha \dfrac{N_O}{N_N}, ~~~~ \alpha \neq 1.
    $$
    (6 points) Calculate $\alpha$.

  \end{enumerate}

\end{document}
\textbf{(10 points)}