\documentclass[fleqn]{article}
\oddsidemargin 0.0in
\textwidth 6.0in
\thispagestyle{empty}
\usepackage{import}
\usepackage{amsmath}
\usepackage{graphicx}
\usepackage{flexisym}
\usepackage{calligra}
\usepackage{amssymb}
\usepackage{bigints} 
\usepackage[english]{babel}
\usepackage[utf8x]{inputenc}
\usepackage{float}
\usepackage[colorinlistoftodos]{todonotes}


\DeclareMathAlphabet{\mathcalligra}{T1}{calligra}{m}{n}
\DeclareFontShape{T1}{calligra}{m}{n}{<->s*[2.2]callig15}{}
\newcommand{\scriptr}{\mathcalligra{r}\,}
\newcommand{\boldscriptr}{\pmb{\mathcalligra{r}}\,}

\definecolor{hwColor}{HTML}{442020}

\begin{document}

  \begin{titlepage}

    \newcommand{\HRule}{\rule{\linewidth}{0.5mm}}

    \center

    \begin{center}
      \includegraphics[height=11cm, width=11cm]{asu.png}
    \end{center}

    \vline

    \textsc{\LARGE Statistical/Thermal Physics}\\[1.5cm]

    \HRule \\[0.5cm]
    { \huge \bfseries Quiz 11}\\[0.4cm] 
    \HRule \\[1.0cm]

    \textbf{Behnam Amiri}

    \bigbreak

    \textbf{Prof: Michael Treacy}

    \bigbreak

    \textbf{{\large \today}\\[2cm]}

    \vfill

  \end{titlepage}

  By signing my name, I am promising that I did this quiz on my own without any outside help.

  \vspace{0.5cm}

  Name: \textbf{Behnam Amiri}

  \vspace{1cm}

  A 1-liter container at $300 ~ K$ contains an ideal gas of ${}^4He$ atoms at a pressure of $10^{-6} ~ Pa$.
  \begin{enumerate}
    \item (6 points) Estimate the number of ${}^4He$ atoms, $N$ , in the container.

      \textcolor{hwColor}{
        \\
        $
          1 ~ liter = 10^{-3} ~ m^3
          \\
          \\
          \\
          PV=nRT \Longrightarrow n=\dfrac{PV}{RT}=\dfrac{ 10^{-6} ~ Pa \times 10^{-3} ~ m^3 }{8.314 ~ J/mol.K \times 300 ~ K}
          \\
          \\
          \\
          \Longrightarrow n \approx 4 \times 10^{-13} ~~ mole
          \\
          \\
          \\
          \text{Number of Atoms}=n \times N_A= 4 \times 10^{-13} \times 6.022 \times 10^{23}
          \\
          \\
          \\
          \therefore ~~~ \boxed{
            \text{Number of Atoms}=2.4144 \times 10^{11}
          } ~~~~ \checkmark
          \\
        $
      }

    \item (6 points) Estimate the number of quantum volumes in the container, $V/v_Q$, at $300 ~ K$.

      \textcolor{hwColor}{
        \\
        From page 253 we have:
        \\
        \\
        $
          \text{Mass of Helium-4 is } 6.7 \times 10^{-27} ~ kg
          \\
          \\
          v_Q=\bigg( \dfrac{h}{\sqrt{2 \pi m K T}} \bigg)^3
          =\bigg( \dfrac{6.626 \times 10^{-34} ~ J.s}{\sqrt{2 \pi (6.7 \times 10^{-27} ~ kg) (1.381 \times 10^{-23} ~ J/K) (300 ~ K)}} \bigg)^3
          \\
          \\
          \\
          \Longrightarrow v_Q \approxeq 1.2629 \times 10^{-31}
          \\
          \\
          \\
          \therefore ~~~ \boxed{
            \dfrac{V}{v_Q}=\dfrac{10^{-3}}{1.2629 \times 10^{-31}}=8 \times 10^{27}
          } ~~~~ \checkmark
          \\
        $
      }

    \pagebreak

    \item (6 points) Estimate the temperature, $T_c$, at which the quantum volume equals the volume per
    particle, i.e. at what $T_c$ does $v_Q=V/N$?

      \textcolor{hwColor}{
        \\
        $
          v_Q=\bigg( \dfrac{h}{\sqrt{2 \pi m K T}} \bigg)^3=\dfrac{V}{N}
          \\
          \\
          \\
          N h^3=V \bigg( 2 \pi m K T \bigg)^{2/3}
          \\
          \\
          \\
          \left[N h^3\right]^{3/2}=\left[V \bigg( 2 \pi m K T \bigg)^{2/3}\right]^{3/2}
          \\
          \\
          \\
          \left[N h^3\right]^{3/2}=2 \pi m K T V^{3/2}
          \\
          \\
          \\
          \Longrightarrow T=\dfrac{
            \left[N h^3\right]^{3/2}
          }{
            2 \pi m K V^{3/2}
          } ~~~~ \checkmark
          \\
          \\
          \\
          T=\dfrac{
            \left[(2.4144 \times 10^{11}) (6.626 \times 10^{-34})^3 \right]^{3/2}
          }{
            2 \pi (6.7 \times 10^{-27}) (1.381 \times 10^{-23}) (10^{-3})^{3/2}
          }
          \\
          \\
          \\
          T_c=3.2018 \times 10^{-80} ~ K
          \\
        $
      }

    \item (6 points) Assuming that Boltzmann statistics apply, estimate the partition function for the
    translational motions, $Z_{tr}$, at temperature $T_c$.
  
      \textcolor{hwColor}{
        \\
        Since we are dealing with a container, we are in 3-D dimensional.
        \\
        $
          E=P.E+K.E=\dfrac{p^2}{2m}
          \\
          \\
          Z_{tr}=\dfrac{1}{h^3} \bigints\limits_{0}^{\infty} e^{-\beta E} d^3
          =\dfrac{4 \pi V}{h^3} \bigints\limits_{0}^{\infty} e^{-\beta \dfrac{p^2}{2m}} d^3
          =\dfrac{4 \pi V}{h^3} \dfrac{\sqrt{3/2}}{2 (\dfrac{\beta}{2m})^{3/2}}
          \\
          \\
          \\
          \therefore ~~~ \boxed{
            Z_{tr}=V \bigg( \dfrac{2 \pi m K_B T}{h^2} \bigg)^{3/2}
          } ~~~~ \checkmark
          \\
          \\
          \\
          Z_{tr}=10^{-3} \bigg( \dfrac{2 \pi (6.7 \times 10^{-27}) (1.381 \times 10^{-23}) (300)}{(6.626 \times 10^{-34})^2} \bigg)^{3/2}
          \Longrightarrow Z_{tr}=7.9177 \times 10^{27}
        $
      }

    \pagebreak

    \item (6 points) Estimate the Helmholtz free energy, $F$ , of the whole system at $T_c$. Give your answer
    in $eV$.

      \textcolor{hwColor}{
        \\
        From page 333 we have:
        \\
        \\
        $
          F=-K T \ln Z=-K T \ln \bigg( V \bigg( \dfrac{2 \pi m K_B T}{h^2} \bigg)^{3/2} \bigg)
          \\
          \\
          \\
          =-(8.617 \times 10^{-5} ~ eV/K)(300 ~ K) \ln(7.9177 \times 10^{27})
          \\
          \\
          \\
          \therefore ~~~ \boxed{
            F \approxeq -1.660 
          } ~~~~ \checkmark
        $
      }

  \end{enumerate}

\end{document}
