\documentclass[fleqn]{article}
\oddsidemargin 0.0in
\textwidth 6.0in
\thispagestyle{empty}
\usepackage{import}
\usepackage{amsmath}
\usepackage{graphicx}
\usepackage{flexisym}
\usepackage{calligra}
\usepackage{amssymb}
\usepackage{bigints} 
\usepackage[english]{babel}
\usepackage[utf8x]{inputenc}
\usepackage{float}
\usepackage[colorinlistoftodos]{todonotes}


\DeclareMathAlphabet{\mathcalligra}{T1}{calligra}{m}{n}
\DeclareFontShape{T1}{calligra}{m}{n}{<->s*[2.2]callig15}{}
\newcommand{\scriptr}{\mathcalligra{r}\,}
\newcommand{\boldscriptr}{\pmb{\mathcalligra{r}}\,}

\definecolor{hwColor}{HTML}{442020}

\begin{document}

  \begin{titlepage}

    \newcommand{\HRule}{\rule{\linewidth}{0.5mm}}

    \center

    \begin{center}
      \includegraphics[height=11cm, width=11cm]{asu.png}
    \end{center}

    \vline

    \textsc{\LARGE Statistical/Thermal Physics}\\[1.5cm]

    \HRule \\[0.5cm]
    { \huge \bfseries Quiz 11}\\[0.4cm] 
    \HRule \\[1.0cm]

    \textbf{Behnam Amiri}

    \bigbreak

    \textbf{Prof: Michael Treacy}

    \bigbreak

    \textbf{{\large \today}\\[2cm]}

    \vfill

  \end{titlepage}

  By signing my name, I am promising that I did this quiz on my own without any outside help.

  \vspace{0.5cm}

  Name: \textbf{Behnam Amiri}

  \vspace{1cm}

  A 1-liter container at $300 ~ K$ contains an ideal gas of 4He atoms at a pressure of $10^{-6} ~ Pa$.
  \begin{enumerate}
    \item (6 points) Estimate the number of 4He atoms, $N$ , in the container.

      % \textcolor{hwColor}{
      %   \\
      % }

    \item (6 points) Estimate the number of quantum volumes in the container, $V/v_Q$, at $300 ~ K$.

      % \textcolor{hwColor}{
      %   \\
      % }

    \item (6 points) Estimate the temperature, $T_c$, at which the quantum volume equals the volume per
    particle, i.e. at what $T_c$ does $v_Q=V/N$?

      % \textcolor{hwColor}{
      %   \\
      % }

    \item (6 points) Assuming that Boltzmann statistics apply, estimate the partition function for the
    translational motions, $Z_tr$, at temperature $T_c$.
  
      % \textcolor{hwColor}{
      %   \\
      % }

    \item (6 points) Estimate the Helmholtz free energy, $F$ , of the whole system at $T_c$. Give your answer
    in $eV$.

      % \textcolor{hwColor}{
      %   \\
      % }

  \end{enumerate}

\end{document}
