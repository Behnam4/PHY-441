\documentclass[fleqn]{article}
\oddsidemargin 0.0in
\textwidth 6.0in
\thispagestyle{empty}
\usepackage{import}
\usepackage{amsmath}
\usepackage{graphicx}
\usepackage{flexisym}
\usepackage{calligra}
\usepackage{amssymb}
\usepackage{bigints} 
\usepackage[english]{babel}
\usepackage[utf8x]{inputenc}
\usepackage{float}
\usepackage[colorinlistoftodos]{todonotes}


\DeclareMathAlphabet{\mathcalligra}{T1}{calligra}{m}{n}
\DeclareFontShape{T1}{calligra}{m}{n}{<->s*[2.2]callig15}{}
\newcommand{\scriptr}{\mathcalligra{r}\,}
\newcommand{\boldscriptr}{\pmb{\mathcalligra{r}}\,}

\definecolor{hwColor}{HTML}{442020}

\begin{document}

  \begin{titlepage}

    \newcommand{\HRule}{\rule{\linewidth}{0.5mm}}

    \center

    \begin{center}
      \includegraphics[height=11cm, width=11cm]{asu.png}
    \end{center}

    \vline

    \textsc{\LARGE Statistical/Thermal Physics}\\[1.5cm]

    \HRule \\[0.5cm]
    { \huge \bfseries Quiz 4}\\[0.4cm] 
    \HRule \\[1.0cm]

    \textbf{Behnam Amiri}

    \bigbreak

    \textbf{Prof: Michael Treacy}

    \bigbreak

    \textbf{{\large \today}\\[2cm]}

    \vfill

  \end{titlepage}

  By signing my name, I am promising that I did this quiz on my own without any outside help.

  \vspace{0.5cm}

  Name: \textbf{Behnam Amiri}

  \vspace{1cm}

  The Sackur-Tetrode equation for the entropy, $S$, of a monatomic ideal gas is
  $$
    S(N,V,U)=N k_B \left[
      ln \bigg( \dfrac{V}{N} \bigg( \dfrac{4 \pi m U}{3 N h^2} \bigg)^{3/2}\bigg)+\dfrac{5}{2}
    \right].
  $$
  $k_B$ is Boltzmann's constant; $N$ is the number of atoms in the volume $V$; m is the atomic mass; $U$
  is the internal energy of the gas, and $h$ is Planck's constant.

  \begin{enumerate}
    \item Use the expression for $U(T)$, given by the equipartition theorem for a monatomic gas, to form
    a new equation for the entropy, $S(N,V,T)$, with $U$ eliminated.

      % \textcolor{hwColor}{
      %   \\
      % }

    \item If the ideal monatomic gas changes from state 1, $(N,V_1,T_1)$ to state 2, $(N,V_2,T_2) $while 
    conserving N, use the previous result to show that the entropy change is given by a formula of the type,
    $$
      \Delta S=S_2-S_1=A(N) ln \bigg( \dfrac{V_2}{V_1} \bigg)+B(N) ln \bigg( \dfrac{T_2}{T_1} \bigg)
    $$
    and determine the coefficients $A(N)$ and $B(N)$.

      % \textcolor{hwColor}{
      %   \\
      % }

    \item If this transition from state 1 to state 2 is adiabatic, use the adiabatic relationship between $V$
    and $T$ to determine the change in entropy, using the equation you derived in part (2) above.

      % \textcolor{hwColor}{
      %   \\
      % }

  \end{enumerate}

\end{document}
