\documentclass[fleqn]{article}
\oddsidemargin 0.0in
\textwidth 6.0in
\thispagestyle{empty}
\usepackage{import}
\usepackage{amsmath}
\usepackage{graphicx}
\usepackage{flexisym}
\usepackage{calligra}
\usepackage{amssymb}
\usepackage{bigints} 
\usepackage[english]{babel}
\usepackage[utf8x]{inputenc}
\usepackage{float}
\usepackage[colorinlistoftodos]{todonotes}


\DeclareMathAlphabet{\mathcalligra}{T1}{calligra}{m}{n}
\DeclareFontShape{T1}{calligra}{m}{n}{<->s*[2.2]callig15}{}
\newcommand{\scriptr}{\mathcalligra{r}\,}
\newcommand{\boldscriptr}{\pmb{\mathcalligra{r}}\,}

\definecolor{hwColor}{HTML}{442020}

\begin{document}

  \begin{titlepage}

    \newcommand{\HRule}{\rule{\linewidth}{0.5mm}}

    \center

    \begin{center}
      \includegraphics[height=11cm, width=11cm]{asu.png}
    \end{center}

    \vline

    \textsc{\LARGE Statistical/Thermal Physics}\\[1.5cm]

    \HRule \\[0.5cm]
    { \huge \bfseries Quiz 4}\\[0.4cm] 
    \HRule \\[1.0cm]

    \textbf{Behnam Amiri}

    \bigbreak

    \textbf{Prof: Michael Treacy}

    \bigbreak

    \textbf{{\large \today}\\[2cm]}

    \vfill

  \end{titlepage}

  By signing my name, I am promising that I did this quiz on my own without any outside help.

  \vspace{0.5cm}

  Name: \textbf{Behnam Amiri}

  \vspace{1cm}

  The Sackur-Tetrode equation for the entropy, $S$, of a monatomic ideal gas is
  $$
    S(N,V,U)=N k_B \left[
      ln \bigg( \dfrac{V}{N} \bigg( \dfrac{4 \pi m U}{3 N h^2} \bigg)^{3/2}\bigg)+\dfrac{5}{2}
    \right].
  $$
  $k_B$ is Boltzmann's constant; $N$ is the number of atoms in the volume $V$; m is the atomic mass; $U$
  is the internal energy of the gas, and $h$ is Planck's constant.

  \begin{enumerate}
    \item Use the expression for $U(T)$, given by the equipartition theorem for a monatomic gas, to form
    a new equation for the entropy, $S(N,V,T)$, with $U$ eliminated.

      \textcolor{hwColor}{
        \\
        If a system contains $N$ molecules, each with $f$ degrees of freedom, and there are no other 
        temperature-dependent forms of energy, then its total thermal energy is 
        $U_{thermal}=N.f.\dfrac{1}{2} kT$. In a gas of monatomic molecules like Helium only translational 
        motions counts, so each molecule has three degree of freedom, that is, $f=3$ Therefore, $U(T)$ for 
        a monatomic gas is:
        \\
        \\
        $
          U(T)=N.3.\dfrac{1}{2}kT=\dfrac{3}{2} NkT
          \\
          \\
          \begin{cases}
            PV=nRT
            \\
            \\
            PV=NKT
          \end{cases} \Longrightarrow nRT=NkT \Longrightarrow \boxed{N=\dfrac{nR}{k}}
          \\
          \\
          \\
          U(T)=\dfrac{3}{2} NkT=\dfrac{3}{2} \bigg( \dfrac{nR}{k} \bigg) kT
          \Longrightarrow \boxed{
            U(T)=\dfrac{3}{2} nRT
          }
          \\
          \\
          \\
          S(N,V,U)=N k_B \left[
            ln \bigg( \dfrac{V}{N} \bigg( \dfrac{4 \pi m U}{3 N h^2} \bigg)^{3/2}\bigg)+\dfrac{5}{2}
          \right]
          =N k_B \left[
            ln \bigg( \dfrac{V}{N} \bigg( \dfrac{4 \pi m \bigg( \dfrac{3}{2} nRT \bigg)}{3 N h^2} \bigg)^{3/2}\bigg)+\dfrac{5}{2}
          \right]
          \\
          \\
          \\
          \therefore ~~~ \boxed{
            S(N,V,T)=N k_B \left[
              ln \bigg( \dfrac{V}{N} \bigg( \dfrac{2 \pi m n R T}{N h^2} \bigg)^{3/2}\bigg)+\dfrac{5}{2}
            \right]
          } ~~~~ \checkmark
          \\
          \\
        $
      }

    \pagebreak

    \item If the ideal monatomic gas changes from state 1, $(N,V_1,T_1)$ to state 2, $(N,V_2,T_2)$ while 
    conserving N, use the previous result to show that the entropy change is given by a formula of the type,
    $$
      \Delta S=S_2-S_1=A(N) ln \bigg( \dfrac{V_2}{V_1} \bigg)+B(N) ln \bigg( \dfrac{T_2}{T_1} \bigg)
    $$
    and determine the coefficients $A(N)$ and $B(N)$.

      \textcolor{hwColor}{
        \\
        Let's start off with simplifying the Natural logarithm value
        \\
        \\
        $
          ln \bigg( \dfrac{V}{N} \bigg( \dfrac{2 \pi m n R T}{N h^2} \bigg)^{3/2}\bigg)
          =ln \bigg( \dfrac{V}{N} \bigg)+ln \bigg( \dfrac{2 \pi m n R T}{N h^2} \bigg)^{3/2}
          =ln V-ln N+\dfrac{3}{2} ln \bigg( \dfrac{2 \pi m n R T}{N h^2} \bigg)
          \\
          \\
          \\
          =ln V-ln N+\dfrac{3}{2} \left[
            ln \bigg(2 \pi m n R T \bigg)-ln \bigg( N h^2 \bigg)
          \right]
          =ln V-ln N+\dfrac{3}{2} \left[
            ln \bigg( T \bigg)+ln \bigg(2 \pi m n R \bigg)-ln \bigg( N h^2 \bigg)
          \right]
          \\
          \\
          \\
          =ln V-ln N+\dfrac{3}{2} \left[
            ln \bigg( T \bigg)+ln \bigg( \dfrac{2 \pi m n R}{N h^2} \bigg)
          \right]
          =ln V-ln N+\dfrac{3}{2} ln \bigg( T \bigg)+\dfrac{3}{2} ln \bigg( \dfrac{2 \pi m n R}{N h^2} \bigg)
          \\
          \\
        $
        We are told that $N$ is conserbed therefore $-ln N+\dfrac{3}{2} ln \bigg( \dfrac{2 \pi m n R}{N h^2} \bigg)$ gives a constant
        value, hence:
        \\
        \\
        $
          S(N,V,T)=N k_B \left[
            ln V+\dfrac{3}{2} ln \bigg( T \bigg)+\text{constant}+\dfrac{5}{2}
          \right]
          \\
          \\
          \\
          \therefore ~~~ \boxed{
            S(N,V,T)=N k_B \left[
              ln V+\dfrac{3}{2} ln \bigg( T \bigg)+\text{constant}
            \right]
          } ~~~~ \checkmark
          \\
          \\
          \\
          \\
          \\
          \\
          \Delta S=S_2(N,V_2,T_2)-S_1(N,V_1,T_1)
          =N k_B \left[
            ln V_2+\dfrac{3}{2} ln \bigg( T_2 \bigg)+\text{constant}
          \right]-N k_B \left[
            ln V_1+\dfrac{3}{2} ln \bigg( T_1 \bigg)+\text{constant}
          \right]
          \\
          \\
          \\
          =N k_B \left[
            \left[ ln V_2+\dfrac{3}{2} ln \bigg( T_2 \bigg)+\text{constant} \right]
            -
            \left[ ln V_1+\dfrac{3}{2} ln \bigg( T_1 \bigg)+\text{constant} \right]
          \right]
          \\
          \\
          \\
          =N k_B \left[
            ln V_2+\dfrac{3}{2} ln \bigg( T_2 \bigg)+\text{constant}
            -ln V_1-\dfrac{3}{2} ln \bigg( T_1 \bigg)-\text{constant}
          \right]
          \\
          \\
          \\
          =N k_B \left[
            ln \bigg( \dfrac{V_2}{V_1} \bigg)+\dfrac{3}{2} ln \bigg( \dfrac{T_2}{T_1} \bigg)
          \right]
          \\
          \\
          \\
          \Delta S=N k_B ~ ln \bigg( \dfrac{V_2}{V_1} \bigg)+N k_B ~ \dfrac{3}{2} ln \bigg( \dfrac{T_2}{T_1} \bigg)
          \\
          \\
          \\
          \begin{cases}
            A(N)=N k_B
            \\
            \\
            B(N)=N k_B ~ \dfrac{3}{2}
          \end{cases} 
          \Longrightarrow 
          \boxed{
            \Delta S=A(N) ~ ln \bigg( \dfrac{V_2}{V_1} \bigg)+B(N) ln \bigg( \dfrac{T_2}{T_1} \bigg)
          } ~~~~ \checkmark
          \\
          \\
        $
      }

    \item If this transition from state 1 to state 2 is adiabatic, use the adiabatic relationship between $V$
    and $T$ to determine the change in entropy, using the equation you derived in part (2) above.

      \textcolor{hwColor}{
        \\
        In page 26 of the textbook we learned that for an adiabatic process $V^{\gamma} P=constant=C_1$.
        \\
        \\
        $
          PV=nRT \Longrightarrow \dfrac{PV}{T}=nR=\text{constant}=C_2
          \\
          \\
          \\
          \therefore ~~~ \boxed{
            P=\dfrac{T C_2}{V}
          } ~~~~ \checkmark
          \\
          \\
          \\
          \therefore ~~~ V^{\gamma} P= V^{\gamma} \dfrac{T C_2}{V}=C_1
          \Longrightarrow \boxed{V^{\gamma-1} T=C} ~~~~ \checkmark ~~ \text{This result is true for an adiabatic process}
          \\
          \\
          \\
          V^{\gamma-1}_1 T_1=V^{\gamma-1}_2 T_2 \Longrightarrow \bigg( \dfrac{V_1}{V_2} \bigg)^{\gamma-1}=\dfrac{T_2}{T_1}
          \\
          \\
          \\
          \Delta S=A(N) ~ ln \bigg( \dfrac{V_2}{V_1} \bigg)+B(N) ln \bigg( \dfrac{T_2}{T_1} \bigg)
          =A(N) ~ ln \bigg( \dfrac{V_2}{V_1} \bigg)+B(N) ln \bigg( \dfrac{V_1}{V_2} \bigg)^{\gamma-1}
          \\
          \\
          \\
          =A(N) ~ ln \bigg( \dfrac{V_2}{V_1} \bigg)+B(N) \bigg( \gamma-1 \bigg) ln \bigg( \dfrac{V_1}{V_2} \bigg)
          =A(N) ~ ln \bigg( \dfrac{V_2}{V_1} \bigg)+B(N) \bigg( 1-\gamma \bigg) ln \bigg( \dfrac{V_2}{V_1} \bigg)
          \\
          \\
          \\
          \therefore ~~~ \boxed{
            \Delta S=ln \bigg( \dfrac{V_2}{V_1} \bigg) \left[A(N)+B(N) \bigg( 1-\gamma \bigg)\right]
          }
          \\
          \\
          \\
        $
        For a monatomic gas $\gamma=\dfrac{5}{3}$ therefore:
        \\
        \\
        $
          \Delta S=ln \bigg( \dfrac{V_2}{V_1} \bigg) \left[N k_B+N k_B ~ \dfrac{3}{2} \bigg( 1-\dfrac{5}{3} \bigg)\right]
          =ln \bigg( \dfrac{V_2}{V_1} \bigg) \left[N k_B+N k_B ~ \dfrac{3}{2} \bigg( -\dfrac{2}{3} \bigg)\right]
          \\
          \\
          \\
          =ln \bigg( \dfrac{V_2}{V_1} \bigg) \left[N k_B-N k_B\right]
          \\
          \\
          \\
          \therefore ~~~ \boxed{
            \Delta S=0
          } ~~~~ \checkmark
        $
      }

  \end{enumerate}

\end{document}
