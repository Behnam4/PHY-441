\documentclass[fleqn]{article}
\oddsidemargin 0.0in
\textwidth 6.0in
\thispagestyle{empty}
\usepackage{import}
\usepackage{amsmath}
\usepackage{graphicx}
\usepackage{flexisym}
\usepackage{calligra}
\usepackage{amssymb}
\usepackage{bigints} 
\usepackage[english]{babel}
\usepackage[utf8x]{inputenc}
\usepackage{float}
\usepackage[colorinlistoftodos]{todonotes}


\DeclareMathAlphabet{\mathcalligra}{T1}{calligra}{m}{n}
\DeclareFontShape{T1}{calligra}{m}{n}{<->s*[2.2]callig15}{}
\newcommand{\scriptr}{\mathcalligra{r}\,}
\newcommand{\boldscriptr}{\pmb{\mathcalligra{r}}\,}

\definecolor{hwColor}{HTML}{442020}

\begin{document}

  \begin{titlepage}

    \newcommand{\HRule}{\rule{\linewidth}{0.5mm}}

    \center

    \begin{center}
      \includegraphics[height=11cm, width=11cm]{asu.png}
    \end{center}

    \vline

    \textsc{\LARGE Statistical/Thermal Physics}\\[1.5cm]

    \HRule \\[0.5cm]
    { \huge \bfseries Quiz 5}\\[0.4cm] 
    \HRule \\[1.0cm]

    \textbf{Behnam Amiri}

    \bigbreak

    \textbf{Prof: Michael Treacy}

    \bigbreak

    \textbf{{\large \today}\\[2cm]}

    \vfill

  \end{titlepage}

  By signing my name, I am promising that I did this quiz on my own without any outside help.

  \vspace{0.5cm}

  Name: \textbf{Behnam Amiri}

  \vspace{1cm}

  One mole of helium gas (He is monatomic) is heated from $300K$ to $400K$ inside an insulated, leak-free 
  cylindrical container with a frictionless moveable piston.

  \begin{enumerate}
    \item Calculate the heat added to the gas (as $Q/R$) if the volume is held constant.

      \textcolor{hwColor}{
        \\
        The first law of thermodynamics states that $\Delta U=Q+W$. Also, we learned that $W=-P \Delta V$.
        \\
        \\
        $
          Q=\Delta U+P \Delta V, ~~ \text{since the volume is held constant then } \Delta V=0.
          \\
          \\
          \begin{cases}
            Q=\Delta U+0
            \\
            \\
            C_V=\dfrac{Q}{\Delta T}
          \end{cases} \Longrightarrow C_V=\bigg( \dfrac{\partial U}{\partial T} \bigg)_V
          \\
          \\
          \\
          \text{For a monatomic ideal gas } \Delta U=\dfrac{3}{2} nR \Delta T
          \\
          \\
          \\
          \therefore ~~~ Q=\Delta U=n C_V \Delta T=\dfrac{3}{2} nR \Delta T
          \Longrightarrow C_V=\dfrac{3}{2} R ~~~~ \checkmark
          \\
          \\
          \\
          Q=\Delta U=\dfrac{3}{2} \bigg( 1 ~ mol \bigg) R \bigg( 400 ~ k-300 ~ k \bigg)=150R
          \\
          \\
          \\
          \therefore ~~~ \boxed{
            \dfrac{Q}{R}=150 
          } ~~~~ \checkmark
          \\
          \\
        $
      }

    \item Calculate the heat added to the gas (as $Q/R$) if the pressure is held constant.

      \textcolor{hwColor}{
        \\
        For this case the pressure is constant. The first law of thermodynamics states that 
        $\Delta U=Q+W$. Also, we learned that $W=-P \Delta V$.
        \\
        \\
        $
          \begin{cases}
            Q=\Delta U-W
            \\
            \\
            Q=n C_P \Delta T
            \\
            \\
            W=-P \Delta V=-nR \Delta T
          \end{cases}
          \\
          \\
          \\
          \text{For a monatomic ideal gas } \Delta U=\dfrac{3}{2} nR \Delta T
          \\
          \\
          \\
          Q=\dfrac{3}{2} n R \Delta T- \bigg( -n R \Delta T \bigg)=\dfrac{5}{2} n R \Delta T
          \\
          \\
          \\
          Q=\dfrac{5}{2} \bigg( 1 ~ mol \bigg) R \bigg( 400 ~ k-300 ~ k \bigg)=\dfrac{5}{2} R \bigg( 100 \bigg)
          \\
          \\
          \\
          \therefore ~~~ \boxed{
            \dfrac{Q}{R}=250
          } ~~~~ \checkmark
          \\
          \\
        $
      }

    \item Calculate the total entropy increase (as $\Delta S/R$) if the volume is held constant. (Do not evaluate
    logarithms.)

      \textcolor{hwColor}{
        \\
        From problem $2.34$ of the textbook learned that for a monatomic ideal gas, the change in entropy is related to
        the heat inout $Q$ by $\Delta S=\dfrac{Q}{T}$ which is valid for any quasistatic process. Note that we are 
        tiold the volume is contant.
        \\
        \\
        $
          Q=\bigints\limits_{V_i}^{V_f} \Delta S=\bigints\limits_{V_i}^{V_f} \dfrac{NkT}{V} dV=NkT ~ ln \dfrac{V_f}{V_i}
          \Longrightarrow \dfrac{Q}{T}=\Delta S ~~~~ \checkmark
          \\
          \\
          \\
          \Delta S=\bigints \dfrac{1}{T} dQ
          =\bigints \dfrac{1}{T} d\bigg( n C_V T \bigg)
          =\bigints \dfrac{n C_V}{T} dT
          =n C_V \bigints\limits_{300}^{400} \dfrac{1}{T} dT
          =1 \times \dfrac{3}{2}R ~ ln\bigg( \dfrac{400}{300} \bigg)
          \\
          \\
          \\
          \therefore ~~~ \boxed{
            \dfrac{\Delta S}{R}=\dfrac{3}{2} ln\bigg( \dfrac{4}{3} \bigg)
          } ~~~~ \checkmark
          \\
          \\
        $
      }

    \item Calculate the total entropy increase (as $\Delta S/R$) if the pressure is held constant. (Do not evaluate
    logarithms.)

      \textcolor{hwColor}{
        \\
        $
          \Delta S=\bigints \dfrac{1}{T} dQ
          =\bigints \dfrac{1}{T} d\bigg( n C_P T \bigg)
          =\bigints \dfrac{n C_P}{T} dT
          =n C_P \bigints\limits_{300}^{400} \dfrac{1}{T} dT
          =1 \times \dfrac{5}{2}R ~ ln\bigg( \dfrac{400}{300} \bigg)
          \\
          \\
          \\
          \therefore ~~~ \boxed{
            \dfrac{\Delta S}{R}=\dfrac{5}{2} ln\bigg( \dfrac{4}{3} \bigg)
          } ~~~~ \checkmark
          \\
          \\
        $
      }

    \item If I transfer 1 Joule of heat to the gas and the entropy changes by $0.0025 ~ J/K$, estimate the
    temperature of the gas. What assumptions are you making?

      % \textcolor{hwColor}{
      %   \\
      % }

    \item A container has a partition separating one mole of helium from $\dfrac{1}{2}$ mole of helium, both sides at
    the same temperature and pressure. The partition is removed and the two gases mix. What is the entropy of 
    mixing? Give your answer as a multiple of $R$.

      % \textcolor{hwColor}{
      %   \\
      % }

    \item A container has a partition separating one mole of helium from $\dfrac{1}{2}$ mole of neon 
    (Ne is monatomic), both sides at the same temperature and pressure. The partition is removed and 
    the two gases mix. Now what is the entropy of mixing? Give your answer as a multiple of R. (Do not evaluate
    logarithms.)

      % \textcolor{hwColor}{
      %   \\
      % }

  \end{enumerate}

\end{document}
