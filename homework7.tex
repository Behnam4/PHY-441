\documentclass[fleqn]{article}
\oddsidemargin 0.0in
\textwidth 6.0in
\thispagestyle{empty}
\usepackage{import}
\usepackage{amsmath}
\usepackage{graphicx}
\usepackage{flexisym}
\usepackage{calligra}
\usepackage{amssymb}
\usepackage{bigints} 
\usepackage[english]{babel}
\usepackage[utf8x]{inputenc}
\usepackage{float}
\usepackage[colorinlistoftodos]{todonotes}


\DeclareMathAlphabet{\mathcalligra}{T1}{calligra}{m}{n}
\DeclareFontShape{T1}{calligra}{m}{n}{<->s*[2.2]callig15}{}
\newcommand{\scriptr}{\mathcalligra{r}\,}
\newcommand{\boldscriptr}{\pmb{\mathcalligra{r}}\,}

\definecolor{hwColor}{HTML}{442020}

\begin{document}

  \begin{titlepage}

    \newcommand{\HRule}{\rule{\linewidth}{0.5mm}}

    \center

    \begin{center}
      \includegraphics[height=11cm, width=11cm]{asu.png}
    \end{center}

    \vline

    \textsc{\LARGE Statistical/Thermal Physics}\\[1.5cm]

    \HRule \\[0.5cm]
    { \huge \bfseries Homework 7}\\[0.4cm] 
    \HRule \\[1.0cm]

    \textbf{Behnam Amiri}

    \bigbreak

    \textbf{Prof: Michael Treacy}

    \bigbreak

    \textbf{{\large \today}\\[2cm]}

    \vfill

  \end{titlepage}

  \begin{enumerate}
    \item \textbf{3.19} Fill in the missing algebraic steps to derive equations $3.30$, $3.31,$ and $3.33$.

    \begin{center}
      \includegraphics[height=15cm, width=15cm]{319.JPG}
    \end{center}

    \pagebreak

    \item \textbf{3.23} Show that the entropy of a two-state paramagnet, expressed as a function of temperature, 
    is $S=N k \left[ln \bigg( 2 cosh ~ x\bigg)-x ~ tanh ~ x\right]$, where $x=\mu B/kT$. Check that this formula
    has the expected behavior as $T \longrightarrow 0$ and $T \longrightarrow \infty$.
  
    \begin{center}
      \includegraphics[height=15cm, width=15cm]{323.JPG}
    \end{center}

    \pagebreak

    \item \textbf{3.25} In Problem $2.18$ you showed that the multiplicity of an Einstein solid containing 
    $N$ oscillators and $q$ energy units is approximately 
    $$
      \Omega \bigg( N, q \bigg)=\bigg( \dfrac{q+N}{q} \bigg)^q \bigg( \dfrac{q+N}{N} \bigg)^N
    $$
    \begin{enumerate}
      \item Starting with this formula....

      \item Use the result of part (a)...

      \item Invert the relation you found...

      \item Show that, in the limit...

      \item Make a graph (possibly using a computer)...

      \item Derive a more accurate approximation for...

    \end{enumerate}

    \begin{center}
      \includegraphics[height=15cm, width=15cm]{325A.JPG}
    \end{center}

    \pagebreak

    \begin{center}
      \includegraphics[height=15cm, width=15cm]{325A.JPG}
    \end{center}

    \pagebreak

    \item \textbf{3.33} Use the thermodynamics identity to derive the heat capacity formula
    $$
      C_V=T \bigg( \dfrac{\partial S}{\partial T} \bigg)_V,
    $$
    which is occasionally more convenient...

    \begin{center}
      \includegraphics[height=15cm, width=15cm]{333.JPG}
    \end{center}

    \pagebreak

    \item \textbf{3.34} Polymers, like rubber, are made of very long molecules...
    \begin{enumerate}
      \item Find an expression for the entropy of this...

      \item Write down a formula for $L$ in terms of $N$ and $N_R$.

      \item For a one-dimensional system such as this...

      \item Using the thermodynamic identity, you can now express...

      \item Show that when $L << N \ell$, the tension force is directly proportional to $L$ (Hooke's law).

      \item Discuss the dependence of the tension force on temperature. If you increase...

      \item Suppose that you hold a relaxed rubber band in both hands and suddenly stretch it. Would you expect...
    \end{enumerate}

    \begin{center}
      \includegraphics[height=14cm, width=14cm]{334.JPG}
    \end{center}

    \pagebreak

    \item \textbf{3.35} In the text I showed that for an Einstein solid with three oscillators and three units of 
    energy, the chemical potential is $\mu=-\epsilon$...

    \begin{center}
      \includegraphics[height=15cm, width=15cm]{335.JPG}
    \end{center}

    \pagebreak

    \item \textbf{4.1} Recall Problem $1.34$, which concerned an ideal diatomic gas taken around a rectangular
    cycle on a $PV$ diagram. Suppose now that this system is used as a heat engine, to convert the heat added
    into mechanical work.
    \begin{enumerate}
      \item Evaluate the efficiency of this engine for the case $V_2=3V_1, P_2=2P_1$.

      \item Calculate the efficiency of an "ideal" engine operating between the same temperature extremes.

    \end{enumerate}

    \begin{center}
      \includegraphics[height=15cm, width=15cm]{41.JPG}
    \end{center}

  \end{enumerate}

\end{document}
