\documentclass[fleqn]{article}
\oddsidemargin 0.0in
\textwidth 6.0in
\thispagestyle{empty}
\usepackage{import}
\usepackage{amsmath}
\usepackage{graphicx}
\usepackage{flexisym}
\usepackage{calligra}
\usepackage{amssymb}
\usepackage{bigints} 
\usepackage[english]{babel}
\usepackage[utf8x]{inputenc}
\usepackage{float}
\usepackage[colorinlistoftodos]{todonotes}


\DeclareMathAlphabet{\mathcalligra}{T1}{calligra}{m}{n}
\DeclareFontShape{T1}{calligra}{m}{n}{<->s*[2.2]callig15}{}
\newcommand{\scriptr}{\mathcalligra{r}\,}
\newcommand{\boldscriptr}{\pmb{\mathcalligra{r}}\,}

\definecolor{hwColor}{HTML}{442020}

\begin{document}

  \begin{titlepage}

    \newcommand{\HRule}{\rule{\linewidth}{0.5mm}}

    \center

    \begin{center}
      \includegraphics[height=11cm, width=11cm]{asu.png}
    \end{center}

    \vline

    \textsc{\LARGE Statistical/Thermal Physics}\\[1.5cm]

    \HRule \\[0.5cm]
    { \huge \bfseries Homework 1}\\[0.4cm] 
    \HRule \\[1.0cm]

    \textbf{Behnam Amiri}

    \bigbreak

    \textbf{Prof: Michael Treacy}

    \bigbreak

    \textbf{{\large \today}\\[2cm]}

    \vfill

  \end{titlepage}

  \begin{enumerate}
    \item \textbf{1.9} What is the volume of one mole of air, at room temperature and 1 atm pressure?
    
        % \textcolor{hwColor}{
        %   \\
        % }

    \item \textbf{1.15} Estimate the average temperature of the air inside a hot-air balloon (see Figure 1.1). Assume 
    that the total mass of the unfilled balloon and payload is $500 ~ kg$. What is the mass of the air inside the balloon?

        % \textcolor{hwColor}{
        %   \\
        % }
    
    \item \textbf{1.17} Even at low density, real gases don't quite obey the ideal gas law. A systematic way to account for deviations 
    from ideal behavior is the \textbf{virial expansion},
    $$
      PV=n ~ RT \bigg(1+\dfrac{B(T)}{(V/n)}+\dfrac{C(T)}{(V/n)^2}+...\bigg),
    $$
    where the functions $B(T), C(T)$, and so on are called \textbf{virial coefficients}... 
    \begin{enumerate}
      \item For each temperature in the table, compute the second term in the virial equation, $B(T)/(V/n),$ for nitrogen at atmospheric
      pressure. Discuss the validity of the ideal gas law under these conditions.

        % \textcolor{hwColor}{
        %   \\
        % }

      \item Think about the forces between molecules, and explain why we might expect $B(T)$ to be negative at low temperatures but positive 
      at high temperatures.

        % \textcolor{hwColor}{
        %   \\
        % }

      \item Any proposed relation between $P, V,$ ans $T$, like the ideal gas law or the virial equations, is called an \textbf{equation of state}.
      Another famous equation of state, which is qualitatively accurate even for dense fluids, is the \textbf{van der Waals equation},
      $$
        \bigg(P+\dfrac{an^2}{V^2}\bigg) \bigg(V-nb\bigg)=nRT,
      $$
      where $a$ and $b$ are constants that depend on the type of gas. Calculate the second and third virial coefficients (B and C) for a gas
      obeying the van der Waals equation, in terms of a and b. (Hint: The bionomial expansion says that 
      $(1+x)^p \approx 1+px+\dfrac{1}{2} p(p-1) x^2$, provided that $|px|<<1$. Apply this approximation to the quantity 
      $\left[1-(nb/V)\right]^{-1}$).

        % \textcolor{hwColor}{
        %   \\
        % }

      \item Plot a graph of the van de Waals prediction for $B(T)$, choosing a and b so as to approximately match the data given above for nitrogen. 
      Discuss the accuracy of the van der Waals equation over this range of conditions. (The van der Waals equation is discussed much further 
      in Section 5.3.)

        % \textcolor{hwColor}{
        %   \\
        % }

    \end{enumerate}
    
    \item \textbf{1.18} Calculate the rms speed of a nitrogen molecule at room temperature.

        % \textcolor{hwColor}{
        %   \\
        % }
    
    \item \textbf{1.22} If you poke a hole in a container full of gas, the gas will start leaking out. In this problem you 
    will make a rough estimate of the rate at which gas escapes through a hole. (This process is called \textbf{effusion}, at least when 
    the hole is sufficiently small.)

      \begin{enumerate}
        \item Consider a small portion (area=A) of the inside wall of a container full of gas. Show that the number of molecules colliding
        with this surface in a time interval $\Delta t$ is $P ~ A ~ \Delta t/(2m \overline{v_x})$, where $P$ is the pressure, $m$ is the 
        average molecular mass, and $\overline{v_x}$ is the average $x$ velocity of those molecules that collide with the wall.

          % \textcolor{hwColor}{
          %   \\
          % }

        \item It is not easy to calculate $\overline{v_x}$, but a good enough approximation is $\overline{v_x}^{1/2}$, where the bar 
        now represents an average over all molecules in the gas. Show that $\overline{v_x}^{1/2}=\sqrt{kT/m}.$

          % \textcolor{hwColor}{
          %   \\
          % }
          
        \item If we now take away this small part of the wall of the container, the molecules that would have collided with it will 
        instead escape through the hole. Assuming that nothing enters through the hole, show that the number $N$ of molecules inside
        the container as a fucntion of time is governed by the differential equation
        $$
          \dfrac{dN}{dt}=-\dfrac{A}{2V} \sqrt{\dfrac{kT}{m}} N
        $$
        Solve this equation (assuming constant temperature) to obtain a formula of the form $N(t)=N(0) ~ e^{-t\tau}$, where $\tau$
        is the "characteristic time" for $N$ (and $P$) to drop by a factor of $e$.

          % \textcolor{hwColor}{
          %   \\
          % }

        \item Calculate the characteristic time for a gas to escape from 1-liter container punctured by a $1-mm^2$ hole.
        
          % \textcolor{hwColor}{
          %   \\
          % }

        \item Your bicycle tire has a slow leak, so that it goes flat within about an hour after being inflated. Roughly how big is the 
        hole? (Use any reasonable estimate for the volume of the tire.)

          % \textcolor{hwColor}{
          %   \\
          % }

        \item In Jules Verne's \emph{Round the Moon}, the escape travelers dispose of a dog's corpse by quickly opening a window,
        tossing it out, and closing the window. Do you think they can do this quickly enough to prevent a significant amount of air
        from escaping? Justify your answer with some rough estimates and calculation.

          % \textcolor{hwColor}{
          %   \\
          % }

      \end{enumerate}

  \end{enumerate}

\end{document}
