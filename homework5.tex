\documentclass[fleqn]{article}
\oddsidemargin 0.0in
\textwidth 6.0in
\thispagestyle{empty}
\usepackage{import}
\usepackage{amsmath}
\usepackage{graphicx}
\usepackage{flexisym}
\usepackage{calligra}
\usepackage{amssymb}
\usepackage{bigints} 
\usepackage[english]{babel}
\usepackage[utf8x]{inputenc}
\usepackage{float}
\usepackage[colorinlistoftodos]{todonotes}


\DeclareMathAlphabet{\mathcalligra}{T1}{calligra}{m}{n}
\DeclareFontShape{T1}{calligra}{m}{n}{<->s*[2.2]callig15}{}
\newcommand{\scriptr}{\mathcalligra{r}\,}
\newcommand{\boldscriptr}{\pmb{\mathcalligra{r}}\,}

\definecolor{hwColor}{HTML}{442020}

\begin{document}

  \begin{titlepage}

    \newcommand{\HRule}{\rule{\linewidth}{0.5mm}}

    \center

    \begin{center}
      \includegraphics[height=11cm, width=11cm]{asu.png}
    \end{center}

    \vline

    \textsc{\LARGE Statistical/Thermal Physics}\\[1.5cm]

    \HRule \\[0.5cm]
    { \huge \bfseries Homework 5}\\[0.4cm] 
    \HRule \\[1.0cm]

    \textbf{Behnam Amiri}

    \bigbreak

    \textbf{Prof: Michael Treacy}

    \bigbreak

    \textbf{{\large \today}\\[2cm]}

    \vfill

  \end{titlepage}

  \begin{enumerate}
    \item \textbf{2.27} Rather than insisting that all the molecules be in the left half of a container, suppose we only require 
    that they be in the leftmost $99\%$ (leaving the remaining $1\%$ completely empty). What is the probability of finding such an
    arrangement if there are 100 molecules in the container? What if there are $10,000$ molecules? What if there are $10^{23}$?

    \begin{center}
      \includegraphics[height=13cm, width=14cm]{1.JPG}
    \end{center}

    \pagebreak

    \item \textbf{2.28} How many possible arrangement are there for a deck of 52 playing cards?..... Is this entrophy significant
    compared to the entrophy associated with arranging thermal energy amoung the molecules in the cards?

    \begin{center}
      \includegraphics[height=13cm, width=14cm]{2.JPG}
    \end{center}

    \pagebreak

    \item \textbf{2.29} Consider a system of two Einstein  solids, with $N_A=300, ~ N_B=200$, and $q_{total}=100$ (as discussed in Section
    2.3). Compute the entrophy of the most likely macrostate...

    \begin{center}
      \includegraphics[height=13cm, width=14cm]{3.JPG}
    \end{center}

    \pagebreak

    \item \textbf{2.30} Consider again the system of two large, identical Einstein solids treated in Problem 2.22.
    \begin{enumerate}
      \item For the case $N=10^{23}$, compute the entrophy of this system...


      \item Compute the entrophy again, assuming that the system is in its most likely macrostate...


      \item Is the issue of time scales really relevant to the entrophy of this system?

      
      \item Suppose that, at a moment when the system is near its most likely macrostate, you suddenly insert a partition between 
      the solids so that they can no longer...

    \end{enumerate}

    \begin{center}
      \includegraphics[height=13cm, width=14cm]{4.JPG}
    \end{center}
  
    \pagebreak

    \item \textbf{2.33} Use the Sackur-Tetrode equation to calculate the entrophy of a mole of argon gas at room 
    temperature and atmospheric pressure. Why is the entrophy greater than that of a mole of helium under the same conditions?

    \begin{center}
      \includegraphics[height=13cm, width=14cm]{5.JPG}
    \end{center}

    \pagebreak

    \item \textbf{2.34} Show that during the quasistatic isothermal expansion of a monatomic ideal gas, the change in entrophy is
    related to the heat input $Q$ by the simple formula
    $$
      \Delta S=\dfrac{Q}{T}
    $$
    In the following chapter, I will prove that this formula is valid for any quasistatic process. Show, however, that it is not valid
    for the free expansion process described above.

    \begin{center}
      \includegraphics[height=13cm, width=14cm]{6.JPG}
    \end{center}

    \pagebreak

    \item \textbf{2.37} Using the same method as in the text....
    $$  
      \Delta S_{mixing}=-Nk \left[x ~ ln\bigg( x \bigg)+\bigg( 1-x \bigg) ~ ln\bigg( 1-x \bigg)\right]
    $$
    Check that this expression reduces to the one given in the text when $x=\dfrac{1}{2}$.

    \begin{center}
      \includegraphics[height=13cm, width=14cm]{7.JPG}
    \end{center}

    \pagebreak

    \item \textbf{2.39} Compute the entrophy of a mole of helium at room temperature and atmospheric pressure, pretending that all 
    the atoms are distinguishable. Compare to the actual entrophy, for indistinguishable atoms, computed in the text.

    \begin{center}
      \includegraphics[height=13cm, width=14cm]{8.JPG}
    \end{center}

  \end{enumerate}

\end{document}
