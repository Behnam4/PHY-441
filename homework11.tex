\documentclass[fleqn]{article}
\oddsidemargin 0.0in
\textwidth 6.0in
\thispagestyle{empty}
\usepackage{import}
\usepackage{amsmath}
\usepackage{graphicx}
\usepackage{flexisym}
\usepackage{calligra}
\usepackage{amssymb}
\usepackage{bigints} 
\usepackage[english]{babel}
\usepackage[utf8x]{inputenc}
\usepackage{float}
\usepackage[colorinlistoftodos]{todonotes}


\DeclareMathAlphabet{\mathcalligra}{T1}{calligra}{m}{n}
\DeclareFontShape{T1}{calligra}{m}{n}{<->s*[2.2]callig15}{}
\newcommand{\scriptr}{\mathcalligra{r}\,}
\newcommand{\boldscriptr}{\pmb{\mathcalligra{r}}\,}

\definecolor{hwColor}{HTML}{442020}

\begin{document}

  \begin{titlepage}

    \newcommand{\HRule}{\rule{\linewidth}{0.5mm}}

    \center

    \begin{center}
      \includegraphics[height=11cm, width=11cm]{asu.png}
    \end{center}

    \vline

    \textsc{\LARGE Statistical/Thermal Physics}\\[1.5cm]

    \HRule \\[0.5cm]
    { \huge \bfseries Homework 11}\\[0.4cm] 
    \HRule \\[1.0cm]

    \textbf{Behnam Amiri}

    \bigbreak

    \textbf{Prof: Michael Treacy}

    \bigbreak

    \textbf{{\large \today}\\[2cm]}

    \vfill

  \end{titlepage}

  \begin{enumerate}
    \item \textbf{5.81} Derive a formula, similar to equation $5.90$, for the shift in the freezing 
    temperature of a dilute solution. Assume that the solid phase is pure solvent, no solute. You should
    find that the shift is negative: The freezing temperature of a solution is \emph{less} than that
    of the pure solvent. Explain in general terms why the shift should be negative.

      \begin{center}
        \includegraphics[height=15cm, width=16cm]{581.JPG}
      \end{center}
    
    \pagebreak

    \item \textbf{5.82} Use the result of the previous problem to calculate the freezing temperature
    of seawater.

      \begin{center}
        \includegraphics[height=15cm, width=16cm]{582.JPG}
      \end{center}

    \pagebreak
    
    \item \textbf{6.2} Prove that the probability of finding an atom in any particular energy level is 
    $P(E)=\dfrac{1}{Z} e^{-F/kT}$, where $F=E-TS$ and the "entropy" of a level is $k$ times logarithm
    of the number of degenerate states for that level.

      \begin{center}
        \includegraphics[height=15cm, width=16cm]{62.JPG}
      \end{center}

    \pagebreak

    \item \textbf{6.3} Consider a hypothetical atom that has two states. a ground state 
    with energy zero and an excited state with energy $2 ~ eV$. Draw a graph...

      \begin{center}
        \includegraphics[height=15cm, width=16cm]{63.JPG}
      \end{center}

    \pagebreak

    \item \textbf{6.5} Imagine a particle that can be in only three states, with energies...
    \begin{enumerate}
      \item Calculate the partition function for this particle.

      \item Calculate the probability for this particle to be in each of the three states.

      \item Because the zero point for measuring energies is arbitrary, we...

    \end{enumerate}

    \begin{center}
      \includegraphics[height=15cm, width=16cm]{65.JPG}
    \end{center}

    \pagebreak

    \item \textbf{6.9} In the numerical example in the text, I calculated only the ratio of the 
    probabilities...
      \begin{enumerate}
        \item Estimate the partition function for a hydrogen atom at $5800 ~ K$...

        \item Show that if all bound states are included in the sum, then the partition function...

        \item When a hydrogen atom is in energy level $n$, the approximate radius of the electron...

      \end{enumerate} 

      \begin{center}
        \includegraphics[height=15cm, width=16cm]{69.JPG}
      \end{center}

    \pagebreak

    \item \textbf{6.23} For a $CO$ molecule, the constant $\epsilon$ is approximately $0.00024 ~ eV$...

      \begin{center}
        \includegraphics[height=15cm, width=16cm]{623.JPG}
      \end{center}

    \pagebreak

    \item \textbf{6.24} For an $O_2$ molecule, the constant $\epsilon$ is approximately $0.00018 ~ eV$. Estimate
    the rotational partition function for an $O_2$ molecule at room temperature.

      \begin{center}
        \includegraphics[height=15cm, width=16cm]{624.JPG}
      \end{center}

  \end{enumerate}

\end{document}
