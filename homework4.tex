\documentclass[fleqn]{article}
\oddsidemargin 0.0in
\textwidth 6.0in
\thispagestyle{empty}
\usepackage{import}
\usepackage{amsmath}
\usepackage{graphicx}
\usepackage{flexisym}
\usepackage{calligra}
\usepackage{amssymb}
\usepackage{bigints} 
\usepackage[english]{babel}
\usepackage[utf8x]{inputenc}
\usepackage{float}
\usepackage[colorinlistoftodos]{todonotes}


\DeclareMathAlphabet{\mathcalligra}{T1}{calligra}{m}{n}
\DeclareFontShape{T1}{calligra}{m}{n}{<->s*[2.2]callig15}{}
\newcommand{\scriptr}{\mathcalligra{r}\,}
\newcommand{\boldscriptr}{\pmb{\mathcalligra{r}}\,}

\definecolor{hwColor}{HTML}{442020}

\begin{document}

  \begin{titlepage}

    \newcommand{\HRule}{\rule{\linewidth}{0.5mm}}

    \center

    \begin{center}
      \includegraphics[height=11cm, width=11cm]{asu.png}
    \end{center}

    \vline

    \textsc{\LARGE Statistical/Thermal Physics}\\[1.5cm]

    \HRule \\[0.5cm]
    { \huge \bfseries Homework 4}\\[0.4cm] 
    \HRule \\[1.0cm]

    \textbf{Behnam Amiri}

    \bigbreak

    \textbf{Prof: Michael Treacy}

    \bigbreak

    \textbf{{\large \today}\\[2cm]}

    \vfill

  \end{titlepage}

  \begin{enumerate}
    \item \textbf{1.50} Consider the combustion of one mole of methane gas:
    $$
      CH_4(gas)+2O_2(gas) \longrightarrow CO_2(gas)+2H_2O(gas)
    $$
    The system is at standard temperature $(298 ~ K)$ and pressure $(10^5 ~ Pa)$ both before and after 
    the reaction.
    \begin{enumerate}
      \item First imagine the process of converting a mole of methane into its elemental constituent (graphite and hydrogen gas).
      Use the data at back of this book to find $\Delta H$ for this process.

          \textcolor{hwColor}{
            \\
            From the table in the textbook:
            \\
            \\
            $
              \Delta H=H_f-H_i=0-\bigg( -74.81 ~ kJ\bigg) \Longrightarrow \boxed{\Delta H=74.81 ~ kJ} ~~~~ \checkmark
            $
            \\
          }

      \item Now imagine forming a mole of $CO_2$ and two moles of water vapor from their elemental constituents. Determine 
      $\Delta H$ for this process.

          \textcolor{hwColor}{
            \\
            $
              \Delta H=H_f-H_i=2\bigg(-241.82 ~ kJ\bigg) \Longrightarrow \boxed{\Delta H=-483.64 ~ kJ} ~~~~ \checkmark
              \\
            $
          }

      \item What is $\Delta H$ for the actual reaction in which methane and oxygen form carbon dioxide and water vapor 
      directly? Explain.

          \textcolor{hwColor}{
            \\
            $
              \Delta H=74.81 ~ kJ-393.51 ~ kJ-483.64 ~ kJ \Longrightarrow \boxed{\Delta H=-802.34 ~ kJ} ~~~~ \checkmark
              \\
            $
          }

      \item How much heat is giving off during this reaction, assuming that no "other" forms of work are done?

          \textcolor{hwColor}{
            \\
            $
              \Delta H=Q+W=Q+0 \Longrightarrow \boxed{Q=-802.34 ~ kJ}
              \\
              \\
            $
            Therefore, the heat given off is precisely equal to the decrease in enthalpy, which is $802.34 ~ kJ$.
            \\
          }

      \item What is the change in the system's energy during this reaction? How would yur answer differ if 
      the $H_2 O$ ended up as liquid water instead of vapor?

          \textcolor{hwColor}{
            \\
            $
              \Delta H=\Delta U+P \Delta V
              \\
              \\
            $
            By the idea gas law in this case, $\Delta V=0$ so $\Delta H=\Delta U=-802.34 ~ kJ$. In case, 
            the $H_2O$ ended up as liquid water instead of vapor then $\Delta H=-2 \times 285.83 ~ kJ=-571.66 ~ kJ$.
            One important fact to noice here is that the volume changes in this case. 
            \\
            \\
            $
              \Delta V=\Delta n \dfrac{RT}{P}=\dfrac{-2RT}{P}
              \\
              \\
            $
            The reason we have some changes in the volume is that initially we have 3 moles and end up with one mole.
            \\
            \\
            $
              \Delta H=\Delta U+P \Delta V 
              \Longrightarrow \Delta U=\Delta H-P\Delta V=-890.36 ~ kJ+2 \bigg( 8.315 ~ J/K\bigg) \bigg( 298 ~ K\bigg)
              \\
              \\
              \therefore ~~~ \boxed{
                \Delta U=-885.40 ~ kJ
              } ~~~~ \checkmark
              \\
              \\
            $ 
          }

      \item The sun has a mass of $2 \times 10^{30} ~ kg$ and gives off energy at a rate of $3.9 \times 10^{26}$
      watts. If the source of the sun's energy were ordinary combustion of a chemical fuel such as methane, about 
      how long could it last?

          \textcolor{hwColor}{
            \\
            The total mass=$16 ~ g+64 ~ g=80 ~ g$.
            \\
            \\
            $
              Q=\Delta H \times \text{The total mass}= 2 \times 10^{30} \times \dfrac{802.34 ~ kJ}{0.080 ~ kg}
              =20.058 \times 10^{36} ~ J
              \\
              \\
              \\
              \Delta t=\dfrac{Q}{P}=\dfrac{20.058 \times 10^{36} ~ J}{3.9 \times 10^{26}}=5.1430 \times 10^{10} ~ seconds
              \\
              \\
              \\
              \therefore ~~~ \boxed{
                \Delta t=1630.8346 ~ Years ~ \approx 1600 ~ Years 
              } ~~~~ \checkmark
              \\
            $
          }

    \end{enumerate}

    \item \textbf{2.12} The natural logarithm function is defined so that $e^{ln ~ x}=x$ for any positive number $x$.
    \begin{enumerate}
      \item Sketch a graph of the natural logarithm function.

          % \textcolor{hwColor}{
          %   \\
          % }

      \item Prove the identities
      $$
        ln ~ ab=ln ~ a+ln ~ b, ~~~~~~ ln ~ a^b=bln ~ a
      $$

          % \textcolor{hwColor}{
          %   \\
          % }

      \item Prove that $\dfrac{d}{dx} ln ~ x=\dfrac{1}{x}$.

          % \textcolor{hwColor}{
          %   \\
          % }

      \item Derive the useful approximation 
      $$
        ln \bigg( x+1 \bigg) \approx x
      $$
      which is valid when $|x| << 1$. Use a calculator to check the accuracy of this approximationfor $x=0.1$ and $x=0.01$

          % \textcolor{hwColor}{
          %   \\
          % }

    \end{enumerate}

    \item \textbf{2.13}  Fun with logarithms.
    \begin{enumerate}
      \item Simplify the expression $e^{a ~ ln ~ b}$. (That is, write it in a way that doesn't involve logarithms.)

          % \textcolor{hwColor}{
          %   \\
          % }

      \item Assuming that $b << a$, prove that $ln \bigg( a+b \bigg) \approx \bigg( ln ~ a\bigg)+\bigg( \dfrac{b}{a}\bigg)$.
      (Hint: Factor out the $a$ from the argument of the algorithm, so that you can apply the approximation of part (d)
      of the previous problem.)

          % \textcolor{hwColor}{
          %   \\
          % }

    \end{enumerate}

    \item \textbf{2.14} Write $e^{10^{23}}$ in the form  $10^x$, for some $x$.

        % \textcolor{hwColor}{
        %   \\
        % }

    \item \textbf{2.16} Suppose you flip 1000 coins.
    \begin{enumerate}
      \item What is the probability of getting exactly 500 heads and 500 tails?

        % \textcolor{hwColor}{
        %   \\
        % }

      \item What is the probability of getting exactly 600 heads and 400 tails?

        % \textcolor{hwColor}{
        %   \\
        % }

    \end{enumerate}

    \item \textbf{2.22} This problem gives an alternative approach to estimating the width of the peak of the multiplicity 
    function for a system of two large Einstein solids.
    \begin{enumerate}
      \item Consider two identical Einstein solids, each with $N$ oscillators, in thermal contact with each other. Suppose
      that the total number...?

        % \textcolor{hwColor}{
        %   \\
        % }

      \item Use the result of Problem 2.18 to find an approximation expression for the total number of microstates for 
      the combined system.

        % \textcolor{hwColor}{
        %   \\
        % }

      \item The most likely macrostate for this system is (of course) the one in which the energy...

        % \textcolor{hwColor}{
        %   \\
        % }

      \item You can get a rough idea of the "sharpness" of multiplicity function by comparing your answers to parts $(b)$
      and $(c)$...

        % \textcolor{hwColor}{
        %   \\
        % }

    \end{enumerate}

    \item \textbf{2.23} Consider a two-state paramagnet with $10^{23}$ elementary dipoles, with the total energy fixed 
    at zero so that exactly half the dipoles point up and half point down.
    \begin{enumerate}
      \item How many microstates are "accessible" to this system?

        % \textcolor{hwColor}{
        %   \\
        % }

      \item Suppose that the microstate of this system changes a billion times per second. How many microstates will it
      explore in ten billion years (the age of the universe)?

        % \textcolor{hwColor}{
        %   \\
        % }

      \item Is it correct to say that, if you wait long enough, a system will eventually be found in every "accessible" microstate?
      Explain your answer, and discuss the meaning of the word "accessible".

        % \textcolor{hwColor}{
        %   \\
        % }
        
    \end{enumerate}
  \end{enumerate}

\end{document}
